%% LyX 2.2.3 created this file.  For more info, see http://www.lyx.org/.
%% Do not edit unless you really know what you are doing.
\documentclass[english]{article}
\usepackage[T1]{fontenc}
\usepackage[latin9]{inputenc}
\usepackage{geometry}
\geometry{verbose,tmargin=3cm,bmargin=3cm,lmargin=3cm,rmargin=3cm}
\usepackage{amsmath}
\usepackage{babel}
\begin{document}

\title{A Simple New Keynesian Model}
\maketitle

\section{Household}

The household problem is
\begin{equation}
\max E_{t}\sum_{j=0}^{\infty}d_{t,t+j}\left(\log\left(C_{t+j}\right)-\frac{\chi}{1+\varphi}N_{t+j}^{1+\varphi}+\upsilon\frac{\left(\min\left(\frac{M_{t}^{d}}{P_{t}},m\right)\right)^{1-\sigma_{q}}}{1-\sigma_{q}}\right)\label{eq:Preferences}
\end{equation}
where $C_{t}$ is consumption, $N_{t}$ is hours worked, $\frac{M_{t}^{d}}{P_{t}}$
is real money balances held by the household and
\[
d_{t,t+j}=\prod_{i=0}^{j}\beta_{t+i-1}.
\]
We assume that
\[
\log\left(\frac{\beta_{t}}{\beta}\right)=\rho_{\beta}\log\left(\frac{\beta_{t-1}}{\beta}\right)+\sigma_{\beta}\varepsilon_{\beta,t}.
\]
The budget constraint is
\begin{equation}
B_{t+1}+P_{t}C_{t}+M_{t}^{d}=R_{t-1}B_{t}+W_{t}N_{t}+T_{t}+M_{t-1}^{d}\label{eq:Budget Constraint}
\end{equation}
where and $B_{t}$ are balances of nominal bonds, $P_{t}$ is the
price of final goods in the home country, $R_{t}$ is the nominal
interest rate on bonds, $W_{t}$ is the wage rate, and $T_{t}$ are
lump-sum profits, transfers, and taxes. The first-order conditions
are
\begin{equation}
\left(C_{t}\right)^{-1}=\Lambda_{t}\label{eq:Marginal Utility of Consumption}
\end{equation}
\begin{equation}
\chi N_{t}^{\varphi}=\Lambda_{t}\frac{W_{t}}{P_{t}}\label{eq:Labor Leisure Tradeoff}
\end{equation}
\begin{equation}
\Lambda_{t}=\beta_{t}R_{t}E_{t}\frac{\Lambda_{t+1}}{\pi_{t+1}}\label{eq:Intertemporal Euler Equation}
\end{equation}
If we assume that real money balances are less than $m$, then
\begin{equation}
\upsilon\left(\frac{M_{t}^{d}}{P_{t}}\right)^{-\sigma_{q}}=\left(\frac{R_{t}-1}{R_{t}}\right)\Lambda_{t}\label{eq:Money Demand}
\end{equation}
where $\pi_{t}\equiv P_{t}/P_{t-1}$.

\section{Retailers}

Final output, $Y_{t}$, is produced by retailers who combine intermediate
goods, $Y_{i,t}$, $i\in\left[0,1\right]$, according to a Dixit-Stiglitz
aggregator, 
\begin{equation}
Y_{t}=\left(\int_{0}^{1}Y_{i,t}^{\frac{\varepsilon-1}{\varepsilon}}di\right)^{\frac{\varepsilon}{\varepsilon-1}},\label{eq:Dixit Stiglitz}
\end{equation}
where $\varepsilon>1$. Profits of the retailers are given by
\begin{equation}
Y_{t}P_{t}-\int_{0}^{1}Y_{i,t}P_{i,t}di=\left(\int_{0}^{1}Y_{i,t}^{\frac{\varepsilon-1}{\varepsilon}}di\right)^{\frac{\varepsilon}{\varepsilon-1}}P_{t}-\int_{0}^{1}Y_{i,t}P_{i,t}di\label{eq:Retailers Profits}
\end{equation}
Notice that the first-order condition with respect to $Y_{i,t}$ is
\begin{eqnarray}
P_{i,t} & = & \left(\int_{0}^{1}Y_{i,t}^{\frac{\varepsilon-1}{\varepsilon}}di\right)^{\frac{1}{\varepsilon-1}}Y_{i,t}^{\frac{-1}{\varepsilon}}P_{t},\label{eq:Retailers FONC}
\end{eqnarray}
which is rearranged (using the definition of $Y_{t}$) to give
\begin{equation}
Y_{i,t}=\left(\frac{P_{i,t}}{P_{t}}\right)^{-\varepsilon}Y_{t}.\label{eq:Demand Curves}
\end{equation}
There is free-entry for retailers, so profits must equal zero. Then
\begin{equation}
P_{t}=\int_{0}^{1}\frac{Y_{i,t}}{Y_{t}}P_{i,t}di=\int_{0}^{1}\left(\frac{P_{i,t}}{P_{t}}\right)^{-\varepsilon}P_{i,t}di=\left(\int_{0}^{1}P_{i,t}^{1-\varepsilon}di\right)^{\frac{1}{1-\varepsilon}}\label{eq:Price Index}
\end{equation}

\section{Intermediate Goods Firms}

Monopolists produce intermediate goods. Monopolists produce with a
linear production function 
\begin{equation}
Y_{i,t}=A_{t}N_{i,t}\label{eq:Production Technology}
\end{equation}
where $A_{t}$ is a productivity shifter that evolves so that
\begin{equation}
\log\left(a_{t}\right)=\mu_{A}\left(1-\rho_{A}\right)+\rho_{A}\log\left(a_{t-1}\right)+\epsilon_{A,t}.\label{eq:Technology Shock}
\end{equation}
where $a_{t}=\frac{A_{t}}{A_{t-1}}$. With the constant-returns-to-scale
production function, real marginal cost, $s_{t}$, is given by the
real wage divided by $A_{t}$, net of a subsidy, $\nu$ paid to firms
to offset monopoly distortions in steady state,

\begin{equation}
\frac{W_{t}/P_{t}}{A_{t}}\left(1-\nu\right)=s_{t}.\label{eq:Real Marginal Cost}
\end{equation}
A fraction $1-\theta$ of monopolists can update their price in period
$t$. The complementary fraction leave their price unchanged from
the previous period. A firm updating its price picks $\tilde{P}_{i,t}$
to solve
\begin{equation}
\max_{\tilde{P}_{i,t}}E_{t}\sum_{j=0}^{\infty}d_{t,t+j}\left(\theta\right)^{j}\Lambda_{t+j}\left(\frac{\tilde{P}_{i,t}}{P_{t+j}}-s_{t+j}\right)\left(\frac{\tilde{P}_{i,t}}{P_{t+j}}\right)^{-\varepsilon}Y_{t+j}.\label{eq:Optimal Pricing Objective}
\end{equation}
The first-order condition is
\begin{equation}
E_{t}\sum_{j=0}^{\infty}d_{t,t+j}\left(\theta\right)^{j}\Lambda_{t+j}\left(\frac{\tilde{P}_{i,t}}{P_{t+j}}-\frac{\varepsilon}{\varepsilon-1}s_{t+j}\right)\left(\frac{P_{t}}{P_{t+j}}\right)^{-\varepsilon}Y_{t+j}=0.\label{eq:Optimal Pricing FONC}
\end{equation}
Note that all monopolists who can update their price will pick the
same price as nothing depends on $i$ in the first-order condition.
Define $\tilde{p}_{t}\equiv\tilde{P}_{i,t}/P_{t}$. Then (\ref{eq:Optimal Pricing FONC})
becomes 
\begin{equation}
E_{t}\sum_{j=0}^{\infty}d_{t,t+j}\left(\theta\right)^{j}\Lambda_{t+j}\left(\frac{\tilde{p}_{t}}{\Pi_{t,t+j}}-\frac{\varepsilon}{\varepsilon-1}s_{t+j}\right)\Pi_{t,t+j}^{\varepsilon}Y_{t+j}=0\label{eq:Optimal Pricing FONC 2}
\end{equation}
where $\Pi_{t,t+j}\equiv P_{t+j}/P_{t}=\prod_{k=1}^{j}\pi_{t+k}$
for $j>0$, and $\Pi_{t,t}\equiv1$. It is sometimes usefule to write
the first-order condition of the firm as
\begin{align*}
F_{1,t} & =\Lambda_{t}Y_{t}+\beta_{t}\theta E_{t}\pi_{t+1}^{\varepsilon-1}F_{1,t+1}\\
F_{2,t} & =\Lambda_{t}Y_{t}s_{t}\frac{\varepsilon}{\varepsilon-1}+\beta_{t}\theta E_{t}\pi_{t+1}^{\varepsilon}F_{1,t+1}\\
F_{1,t}\tilde{p}_{t} & =F_{2,t}
\end{align*}

\section{Government Policy}

To close the model, specify the interest rate so that
\begin{equation}
R_{t}=R\pi_{t}^{\phi_{\pi}}\exp\left(\epsilon_{R,t}\right).\label{eq:Taylor Rule}
\end{equation}
The money supply is set to satisfy money demand, given (\ref{eq:Money Demand}).
Government consumption is set so that
\begin{equation}
\log\left(\frac{G_{t}}{G}\right)=\rho_{G}\log\left(\frac{G_{t-1}}{G}\right)+\epsilon_{G,t}.\label{eq:Evolution of G}
\end{equation}
 The government budget constraint is
\begin{equation}
P_{t}G_{t}+\nu W_{t}N_{t}+R_{t-1}B_{t}-T_{t}=B_{t+1}.\label{eq:Government Budget Constraint}
\end{equation}
Assume that in each period bonds are in zero net supply ($B_{t}=0$).

\section{Price evolution, market clearing, and aggregation}

From (\ref{eq:Price Index}) prices evolve so that
\[
P_{t}=\left(\int_{0}^{1}P_{i,t}^{1-\epsilon}di\right)^{\frac{1}{1-\varepsilon}}=\left(\left(1-\theta\right)\tilde{P}_{t}^{1-\varepsilon}+\theta\int_{0}^{1}P_{i,t-1}^{1-\varepsilon}di\right)^{\frac{1}{1-\varepsilon}}=\left(\left(1-\theta\right)\tilde{P}_{t}^{1-\varepsilon}+\theta P_{t-1}^{1-\varepsilon}\right)^{\frac{1}{1-\varepsilon}}.
\]
Then
\[
1=\left(\left(1-\theta\right)\tilde{p}_{t}^{1-\varepsilon}+\theta\pi_{t}^{\varepsilon-1}\right)^{\frac{1}{1-\varepsilon}},
\]
where $\tilde{p}_{t}\equiv\tilde{P}_{t}/P_{t}$, which means
\begin{equation}
\tilde{p}_{t}=\left(\frac{1-\theta\pi_{t}^{\varepsilon-1}}{1-\theta}\right)^{\frac{1}{1-\varepsilon}}.\label{eq:relationship between ptilde and inflation}
\end{equation}
Market clearing in the labor market implies
\begin{equation}
N_{t}=\int_{0}^{1}N_{i,t}di.\label{eq:Labor-Market Clearning}
\end{equation}
Combining with (\ref{eq:Production Technology})
\[
A_{t}N_{t}=\int_{0}^{1}Y_{i,t}di=\int_{0}^{1}\left(\frac{P_{i,t}}{P_{t}}\right)^{-\varepsilon}Y_{t}di.
\]
Define 
\begin{equation}
p_{t}^{*}\equiv\left(\int_{0}^{1}\left(\frac{P_{i,t}}{P_{t}}\right)^{-\varepsilon}di\right)^{-1}\label{eq:pstar}
\end{equation}
so that
\begin{equation}
Y_{t}=p_{t}^{*}A_{t}N_{t}.\label{eq:Aggregate Production Technology}
\end{equation}
Note that
\begin{equation}
p_{t}^{*}=\left(\left(1-\theta\right)\tilde{p}_{t}^{-\varepsilon}+\theta\int_{0}^{1}\left(\frac{P_{i,t-1}}{P_{t-1}}\frac{P_{t-1}}{P_{t}}\right)^{-\varepsilon}di\right)^{-1}=\left(\left(1-\theta\right)\tilde{p}_{t}^{-\varepsilon}+\theta\pi_{t}^{\varepsilon}\left(p_{t-1}^{*}\right)^{-1}\right)^{-1}.\label{eq:pstar evolution}
\end{equation}
The resource constraint is
\begin{equation}
C_{t}+G_{t}=Y_{t}.\label{eq:Resource Constraint}
\end{equation}

\section{Equilibrium}

The 12 unknown aggregate variables are $C_{t}$, $N_{t}$, $Y_{t}$,
$m_{t}^{d}\equiv M_{t}^{d}/P_{t}$, $w_{t}\equiv W_{t}/P_{t}$, $\pi_{t}$,
$\Lambda_{t}$, $R_{t}$, $s_{t}$, $\tilde{p}_{t}$, $p_{t}^{*}$,
and $T_{t}$. We determine these varibles with equations (\ref{eq:Marginal Utility of Consumption}),
(\ref{eq:Labor Leisure Tradeoff}), (\ref{eq:Intertemporal Euler Equation}),
(\ref{eq:Money Demand}), (\ref{eq:Real Marginal Cost}), (\ref{eq:Optimal Pricing FONC 2}),
(\ref{eq:Taylor Rule}), (\ref{eq:Government Budget Constraint}),
(\ref{eq:relationship between ptilde and inflation}), (\ref{eq:Aggregate Production Technology}),
(\ref{eq:pstar evolution}), and (\ref{eq:Resource Constraint}).
Note that we do not include the household budget constraint, (\ref{eq:Budget Constraint}),
in the set of equilibrium conditions because of Walras' Law. Also,
$T_{t}$ appears only in (\ref{eq:Government Budget Constraint}),
and we assume it is set to satisfy this equation on a per-period basis,
along with our assumption that $B_{t}=0$.\footnote{The assumption that $B_{t}=0$ is stronger than necessary. The path
for $B_{t}$ can be any bounded path, so long as $T_{t}$ makes (\ref{eq:Government Budget Constraint})
hold period-by-period.} As such, we can eliminate $T_{t}$ from the unknown aggregate variables,
along with equation (\ref{eq:Government Budget Constraint}). Similarly,
$m_{t}^{d}$ appears only in equation (\ref{eq:Money Demand}), and
we eliminate both.

The exogenous variables, $A_{t}$ and $G_{t}$, evolve according to
(\ref{eq:Technology Shock}) and (\ref{eq:Evolution of G}). The shocks,
$\epsilon_{A,t}$, $\epsilon_{G,t}$, and $\epsilon_{R,t}$ , are
normally distributed with standard deviations $\sigma_{A}$, $\sigma_{G}$,
and $\sigma_{R}$ , respectively.

\section{Reasonable Parameter Values}
Some reasonable parameter values are:
\begin{align*}
  \beta &= 0.9975\\
  \chi &= 1.25\\
  \varphi &= 1\\
  \sigma_q &= 10\\
  \varepsilon &= 11\\
  \upsilon &= 1-(\varepsilon-1)/\varepsilon \\
  \theta &= 0.65\\
  \theta_\pi &= 1.5\\
\end{align*}
For the shocks, some reasonable parameter values are
\begin{align*}
  \rho_A &= 0.95\\
  \rho_\beta &= 0.8\\
  \rho_g &= 0.8\\
  \sigma_A &= 0.01\\
  \sigma_\beta &= 0.0005\\
  \sigma_g &= 0.01\\
\end{align*}
Note that we assume $\eta_G\equiv G/Y$ (steady state) is 0.2.  Steady state labor should be 1 with the value of $\chi$ above.
\section{Steady State}

Consider a steady state where inflation is equal to target inflation,
$\pi=1$. From (\ref{eq:relationship between ptilde and inflation})
and (\ref{eq:pstar}), $\tilde{p}=p^{*}=1$. From (\ref{eq:Intertemporal Euler Equation}),
$R=a\beta^{-1}$. Assume that $G=\eta_{G}Y$. We will verify that
we can pick the parameter $\chi$ so that $N=1$. Note that in steady
state, $a=\exp\left(\mu_{A}\right)$, so (\ref{eq:Aggregate Production Technology})
implies $y=\frac{Y}{A}=1$. Then $c=\frac{C}{A}=1-\eta_{G}$ and $g=\frac{G}{A}=\eta_{G}$.
From (\ref{eq:Marginal Utility of Consumption}), $\lambda=\frac{\Lambda}{a^{-1}}=\left(\frac{C}{A}\right)^{-1}=c^{-1}$
. Equation (\ref{eq:Optimal Pricing FONC 2}) implies that $s=\frac{\varepsilon-1}{\varepsilon}$,
meaning that (\ref{eq:Real Marginal Cost}) implies that $\frac{W}{AP}=w=s/(1-\nu)$.
Finally, use (\ref{eq:Labor Leisure Tradeoff}) to determine $\chi$.
Note that you could also back out $m^{d}$ from (\ref{eq:Money Demand}).

\section{Efficient Steady State}

Consider a social planner who maximizes (\ref{eq:Preferences}) subject
to (\ref{eq:Production Technology}) and (\ref{eq:Resource Constraint}).
Set $\upsilon$ to zero so that we ignore money demand (this is the
standard thing to do). It is easy to show that
\[
\chi\left(\frac{Y_{t}}{A_{t}}\right)^{\varphi}=\frac{A_{t}}{C_{t}}
\]
In steady state, this condition is equivalent to the steady state
with $\pi=1$ so $p^{*}=1$ and where $\nu$ is set so that $w=1$. 

\section{Linearization}

For any variable $x_{t}$, define $\hat{x}_{t}\equiv\log\left(x_{t}/x\right)$,
where $x$ is the steady state value of $x_{t}$. Note that equations
(\ref{eq:Marginal Utility of Consumption}), (\ref{eq:Labor Leisure Tradeoff}),
(\ref{eq:Intertemporal Euler Equation}), (\ref{eq:Real Marginal Cost}),
(\ref{eq:Taylor Rule}), and (\ref{eq:Aggregate Production Technology}),
are log-linear so that
\begin{equation}
-\hat{c}_{t}=\hat{\lambda}_{t}\label{eq:LL Marginal Utility of Consumption}
\end{equation}
\begin{equation}
\varphi\hat{n}_{t}=\hat{\lambda}_{t}+\hat{w}_{t}\label{eq:LL Labor-Leisure Tradeoff}
\end{equation}
\begin{equation}
\hat{\lambda}_{t}=\hat{\beta}_{t}+\hat{R}_{t}-E_{t}\hat{a}_{t+1}+E_{t}\hat{\lambda}_{t+1}-E_{t}\hat{\pi}_{t+1}\label{eq:LL Intertemporal Euler Equation}
\end{equation}
\begin{equation}
\hat{w}_{t}=\hat{s}_{t}\label{eq:LL Real Marginal Cost}
\end{equation}
\begin{equation}
\hat{R}_{t}=\phi_{\pi}\hat{\pi}_{t}+\epsilon_{R,t}\label{eq:LL Taylor Rule}
\end{equation}
\begin{equation}
\hat{y}_{t}=\hat{p}_{t}^{*}+\hat{N}_{t}.\label{eq:LL Production Technology}
\end{equation}
The resource constraint, (\ref{eq:Resource Constraint}), is linearized
to be
\begin{equation}
\hat{y}_{t}=\left(1-\eta_{G}\right)\hat{c}_{t}+\eta_{G}\hat{g}_{t}.\label{eq:LL Resource Constraint}
\end{equation}
The equation (\ref{eq:relationship between ptilde and inflation})
implies
\begin{equation}
\hat{\tilde{p}}_{t}=\frac{\theta}{1-\theta}\hat{\pi}_{t}.\label{eq:LL relationship between ptilde and inflation}
\end{equation}
The equation (\ref{eq:pstar evolution}) implies
\begin{equation}
-\hat{p}_{t}^{*}=-\varepsilon\left(1-\theta\right)\hat{\tilde{p}}_{t}+\theta\left(\varepsilon\hat{\pi}_{t}-\hat{p}_{t-1}^{*}\right).\label{eq:LL pstar evolution}
\end{equation}
Finally, equation (\ref{eq:Optimal Pricing FONC 2}) can be linearized
in the following way. First, linearize the term
\[
\left(\frac{\tilde{p}_{t}}{\Pi_{t,t+j}}-\frac{\varepsilon}{\varepsilon-1}s_{t+j}\right)\approx\hat{\tilde{p}}_{t}-\hat{\Pi}_{t,t+j}-\hat{s}_{t+j}=\hat{\tilde{p}}_{t}-\sum_{k=1}^{j}\hat{\pi}_{t+k}-\hat{s}_{t+j}
\]
where the equality follow from the definition of $\Pi_{t,t+j}$. Note
that in steady state $1-\frac{\varepsilon}{\varepsilon-1}s=0$, so
(\ref{eq:Optimal Pricing FONC 2}) can be expressed in log-linear
form as 
\[
E_{t}\sum_{j=0}^{\infty}\left(\beta\theta\right)^{j}\left(\hat{\tilde{p}}_{t}-\sum_{k=1}^{j}\hat{\pi}_{t+k}-\hat{s}_{t+j}\right)=0.
\]
Now, write this as
\begin{eqnarray}
\hat{\tilde{p}}_{t} & = & \left(1-\beta\theta\right)E_{t}\sum_{j=0}^{\infty}\left(\beta\theta\right)^{j}\left[\sum_{k=1}^{j}\hat{\pi}_{t+k}+\hat{s}_{t+j}\right]\nonumber \\
 & = & \left(1-\beta\theta\right)\hat{s}_{t}+\left(1-\beta\theta\right)E_{t}\sum_{j=1}^{\infty}\left(\beta\theta\right)^{j}\left[\sum_{k=1}^{j}\hat{\pi}_{t+k}+\hat{s}_{t+j}\right]\nonumber \\
 & = & \left(1-\beta\theta\right)\hat{s}_{t}+\left(1-\beta\theta\right)E_{t}\sum_{j=0}^{\infty}\left(\beta\theta\right)^{j+1}\left[\sum_{k=1}^{j+1}\hat{\pi}_{t+k}+\hat{s}_{t+1+j}\right]\nonumber \\
 & = & \left(1-\beta\theta\right)\hat{s}_{t}+\beta\xi\left(1-\beta\theta\right)E_{t}\sum_{j=0}^{\infty}\left(\beta\theta\right)^{j}\left[\hat{\pi}_{t+1}+\sum_{k=1}^{j}\hat{\pi}_{t+1+k}+\hat{s}_{t+1+j}\right]\nonumber \\
 & = & \left(1-\beta\theta\right)\hat{s}_{t}+\beta\theta E_{t}\left(\hat{\pi}_{t+1}+\hat{\tilde{p}}_{t+1}\right).\label{eq:LL Optimal Pricing Decision}
\end{eqnarray}

\subsection{Simplification}

Sub out $\hat{\lambda}_{t}$ using (\ref{eq:LL Marginal Utility of Consumption}),
sub out $\hat{w}_{t}$ using (\ref{eq:LL Labor-Leisure Tradeoff}),
sub out $\hat{y}_{t}$ using (\ref{eq:LL Resource Constraint}), sub
out $\hat{s}_{t}$ using (\ref{eq:LL Real Marginal Cost}), and sub
out $\hat{\tilde{p}}_{t}$ using (\ref{eq:LL relationship between ptilde and inflation})
to get
\[
-\hat{c}_{t}=\hat{\beta}_{t}+\hat{R}_{t}-E_{t}\hat{a}_{t+1}-E_{t}\hat{c}_{t+1}-E_{t}\hat{\pi}_{t+1}
\]
\[
\hat{R}_{t}=\phi_{\pi}\hat{\pi}_{t}+\epsilon_{R,t}
\]
\[
\left(1-\eta_{G}\right)\hat{c}_{t}+\eta_{G}\hat{g}_{t}=\hat{p}_{t}^{*}+\hat{N}_{t}
\]

\[
\hat{p}_{t}^{*}=\theta\hat{p}_{t-1}^{*}
\]
\[
\hat{\pi}_{t}=\frac{\left(1-\beta\theta\right)\left(1-\theta\right)}{\beta}\left(\varphi\hat{N}_{t}+\hat{c}_{t}\right)+\beta E_{t}\hat{\pi}_{t+1}.
\]
Note that the fourth of these equations implies that $\hat{p}_{t}^{*}\rightarrow0$,
so $p_{t}^{*}\approx1$ and $\hat{p}_{t}^{*}\approx0$. Sub out for
$\hat{N}_{t}$ using the third equation and we have a 3-equation system:
\begin{equation}
-\hat{c}_{t}=\hat{\beta}_{t}+\hat{R}_{t}-E_{t}\hat{a}_{t+1}-E_{t}\hat{c}_{t+1}-E_{t}\hat{\pi}_{t+1},\label{eq:3-Equation NK IS}
\end{equation}

\begin{equation}
\hat{\pi}_{t}=\frac{\left(1-\beta\theta\right)\left(1-\theta\right)}{\beta}\left(\left[\varphi\left(1-\eta_{G}\right)+1\right]\hat{c}_{t}+\varphi\eta_{G}\hat{g}_{t}\right)+\beta E_{t}\hat{\pi}_{t+1},\label{eq:3-Equation NK Phillips}
\end{equation}
\begin{equation}
\hat{R}_{t}=\phi_{\pi}\hat{\pi}_{t}+\epsilon_{R,t}.\label{eq:3-Equation NK Policy Rule}
\end{equation}

\section{Piecewise-linear model}

Let's ignore government spending shocks and monetary policy shocks.
Let's also define $\hat{\eta}_{t}\equiv\hat{\beta}_{t}-E_{t}\hat{a}_{t+1}$.
Then we have the 3-equation model that we have been using.
\begin{equation}
-\hat{c}_{t}=\hat{R}_{t}+\hat{\eta}_{t}-E_{t}\hat{c}_{t+1}-E_{t}\hat{\pi}_{t+1},\label{eq:PL NK IS}
\end{equation}

\begin{equation}
\hat{\pi}_{t}=\frac{\left(1-\beta\theta\right)\left(1-\theta\right)}{\beta}\left(\left[\varphi\left(1-\eta_{G}\right)+1\right]\hat{c}_{t}\right)+\beta E_{t}\hat{\pi}_{t+1},\label{eq:PL NK Phillips}
\end{equation}
\begin{equation}
\hat{R}_{t}=\max\left\{ \log\left(\frac{\beta}{a}\right),\phi_{\pi}\hat{\pi}_{t}\right\} \label{eq:PL Policy Rule}
\end{equation}
We could define an AR(1) process for $\hat{\eta}_{t}$ directly.

\section{Natural Rates}

Define the natural rates of aggregate variables to the (real) prices
and and quantities that would prevail with no nominal rigidities.
All prices are identical because firms are symmetric. So $p_{t}^{*}=1$
and $\tilde{p}_{t}=1$. Equations (\ref{eq:LL Marginal Utility of Consumption})
and (\ref{eq:LL Labor-Leisure Tradeoff}) are unchanged. The intertemporal
Euler equation becomes
\begin{equation}
\hat{\lambda}_{t}=\hat{\beta}_{t}+\hat{r}_{t}^{*}-E_{t}\hat{a}_{t+1}+E_{t}\hat{\lambda}_{t+1}.\label{eq:LL Intertemporal Euler Equation natural}
\end{equation}
where $r_{t}^{*}$ is the natural rate of interest. Equation (\ref{eq:LL Real Marginal Cost})
is unchanged. Aggregate production technology is given by
\begin{equation}
\hat{y}_{t}=\hat{N}_{t}.\label{eq:LL Production Technology natural}
\end{equation}
The resource constraint (\ref{eq:LL Resource Constraint}) is unchanged.
Finally, firm optimality implies
\begin{equation}
\hat{s}_{t}=0\label{eq:LL Optimal Pricing Decision natural}
\end{equation}
The equations determining equilibrium are then
\begin{equation}
-\hat{c}_{t}=\hat{\beta}_{t}+\hat{r}_{t}^{*}-E_{t}\hat{a}_{t+1}-E_{t}\hat{c}_{t+1}\label{eq:Natural IS}
\end{equation}
\begin{equation}
\varphi\left(\left(1-\eta_{G}\right)\hat{c}_{t}+\eta_{G}\hat{g}_{t}\right)=-\hat{c}_{t}\label{eq:Natural Phillips}
\end{equation}
Notice that if $\hat{\pi}_{t}=0$ and $\hat{R}_{t}=\hat{r}_{t}^{*}$,
then (\ref{eq:3-Equation NK IS}) and (\ref{eq:3-Equation NK Phillips})
are the same as (\ref{eq:Natural IS}) and (\ref{eq:Natural Phillips}).
If we assume that we have linearized around the efficient steady state,
then we have the first-order efficient allocations if $R_{t}=r_{t}^{*}$.
Of course, and as is usual, we are ignoring money demand.

\section{Simplified nonlinear model}

First, write the non-linear equations in a form that is scaled by
$A_{t}$.

\[
\chi N_{t}^{\varphi}=c_{t}^{-1}w_{t}
\]
\[
1=\beta_{t}R_{t}E_{t}\frac{c_{t}}{a_{t+1}c_{t+1}\pi_{t+1}}
\]
\[
w_{t}\left(1-\nu\right)=s_{t}
\]
\[
F_{1,t}=c_{t}^{-1}y_{t}+\beta_{t}\theta E_{t}\pi_{t+1}^{\varepsilon-1}F_{1,t+1}
\]
\[
F_{2,t}=c_{t}^{-1}y_{t}s_{t}\frac{\varepsilon}{\varepsilon-1}+\beta_{t}\theta E_{t}\pi_{t+1}^{\varepsilon}F_{1,t+1}
\]
\[
F_{1,t}\tilde{p}_{t}=F_{2,t}
\]
\[
R_{t}=\max\left\{ R\pi_{t}^{\phi_{\pi}}\exp\left(\epsilon_{R,t}\right),1\right\} 
\]
\[
\tilde{p}_{t}=\left(\frac{1-\theta\pi_{t}^{\varepsilon-1}}{1-\theta}\right)^{\frac{1}{1-\varepsilon}}
\]
\[
y_{t}=p_{t}^{*}N_{t}
\]
\[
p_{t}^{*}=\left(\left(1-\theta\right)\tilde{p}_{t}^{-\varepsilon}+\theta\pi_{t}^{\varepsilon}\left(p_{t-1}^{*}\right)^{-1}\right)^{-1}
\]
\[
c_{t}+g_{t}=y_{t}
\]
Next, simplify
\[
1=\beta_{t}R_{t}E_{t}\frac{c_{t}}{a_{t+1}c_{t+1}\pi_{t+1}}
\]
\[
F_{1,t}=c_{t}^{-1}\left(c_{t}+g_{t}\right)+\beta_{t}\theta E_{t}\pi_{t+1}^{\varepsilon-1}F_{1,t+1}
\]
\[
F_{2,t}=\chi\left(c_{t}+g_{t}\right)\left(\frac{c_{t}+g_{t}}{p_{t}^{\ast}}\right)^{\varphi}\left(1-\nu\right)\frac{\varepsilon}{\varepsilon-1}+\beta_{t}\theta E_{t}\pi_{t+1}^{\varepsilon}F_{1,t+1}
\]
\[
F_{1,t}\left(\frac{1-\theta\pi_{t}^{\varepsilon-1}}{1-\theta}\right)^{\frac{1}{1-\varepsilon}}=F_{2,t}
\]
\[
R_{t}=\max\left\{ R\pi_{t}^{\phi_{\pi}}\exp\left(\epsilon_{R,t}\right),1\right\} 
\]
\[
p_{t}^{*}=\left(\left(1-\theta\right)\left(\frac{1-\theta\pi_{t}^{\varepsilon-1}}{1-\theta}\right)^{-\frac{\varepsilon}{1-\varepsilon}}+\theta\pi_{t}^{\varepsilon}\left(p_{t-1}^{*}\right)^{-1}\right)^{-1}
\]

\section{A simplified process for the discount rate}

Assume that $\beta_{t}\in\left\{ \beta^{\ell},\beta\right\} $ and
assume that $\beta$ is an absorbing state. When $\beta_{t}=\beta^{\ell},$
the probability of switching to $\beta$ is $1-p$ . Assume there
are no other shocks. Denote the outcomes when $\beta_{t}=\beta^{\ell}$
by $x_{t}^{\ell}$. Denote the outcomes when $\beta_{t}=\beta$ by
$x_{t}$. We can write the equation characterizing equilibrium as
\[
1=\frac{\beta^{\ell}}{a}\max\left\{ R\left(\pi_{t}^{\ell}\right)^{\phi_{\pi}},1\right\} \left[p\frac{c_{t}^{\ell}}{c_{t+1}^{\ell}\pi_{t+1}^{\ell}}+\left(1-p\right)\frac{c_{t}^{\ell}}{c_{t+1}\pi_{t+1}}\right]
\]
\[
F_{1,t}^{\ell}=\left(c_{t}^{\ell}\right)^{-1}\left(c_{t}^{\ell}+g\right)+\beta^{\ell}\theta\left[p\left(\pi_{t+1}^{\ell}\right)^{\varepsilon-1}F_{1,t+1}^{\ell}+\left(1-p\right)\left(\pi_{t+1}\right)^{\varepsilon-1}F_{1,t+1}\right]
\]
\begin{align*}
F_{1,t}^{\ell}\left(\frac{1-\theta\left(\pi_{t}^{\ell}\right)^{\varepsilon-1}}{1-\theta}\right)^{\frac{1}{1-\varepsilon}}= & \chi\left(c_{t}^{\ell}+g\right)\left(\frac{c_{t}^{\ell}+g}{p_{t}^{\ast}}\right)^{\varphi}\left(1-\nu\right)\frac{\varepsilon}{\varepsilon-1}\\
 & +\beta^{\ell}\theta\left[p\left(\pi_{t+1}^{\ell}\right)^{\varepsilon}F_{1,t}^{\ell}\left(\frac{1-\theta\left(\pi_{t}^{\ell}\right)^{\varepsilon-1}}{1-\theta}\right)^{\frac{1}{1-\varepsilon}}+\left(1-p\right)\left(\pi_{t+1}\right)^{\varepsilon}F_{1,t+1}\left(\frac{1-\theta\pi_{t+1}^{\varepsilon-1}}{1-\theta}\right)^{\frac{1}{1-\varepsilon}}\right]
\end{align*}
\[
p_{t}^{*}=\left(\left(1-\theta\right)\left(\frac{1-\theta\pi_{t}^{\varepsilon-1}}{1-\theta}\right)^{-\frac{\varepsilon}{1-\varepsilon}}+\theta\pi_{t}^{\varepsilon}\left(p_{t-1}^{*}\right)^{-1}\right)^{-1}
\]
One path in any equilibrium is the probability zero event that $\beta_{t}=\beta^{\ell}$
forever. In that case, $p_{t}^{\ast}\rightarrow p^{\ast\ell}$, $\pi_{t}^{\ell}\rightarrow\pi^{\ell}$
, $c_{t}^{\ell}\rightarrow c^{\ell}$, $F_{1,t}^{\ell}\rightarrow F_{1}^{\ell}$
. Note that the model has no state variables other than $p_{t}^{\ast}$,
so the $\pi_{t}$, $c_{t}$, $F_{1,t}$ are functions only of $p_{t}^{\ast}$.
For a given $p^{\ast\ell}$, this is a set of numbers. We can check
to see if there are limit points in the following way. Conjecture
a guess of $\pi^{\ell}$ on the support of $\pi$ (from 0 to a calculable
positive number. The number 1 is generally good enough for a high
$\beta$ because we want ZLB equilibria). Give $\pi^{\ell}$ the fourth
equation gives $p^{\ast\ell}$, which then gives us all of the $x_{t+1}$
variables. The first equation gives us $c^{\ell}$. The second equation
gives us $F_{1}^{\ell}$. The third equation can be checked to see
if it equals zero. If it does, we have an equilibrium. If not, we
do not have an equilibrium. We can check all values of $\pi^{\ell}$,
meaning we can check for existence.
\end{document}
